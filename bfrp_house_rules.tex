\chapter{House Rules}
\section{Characters}

\subsection{PC Creation}
\begin{itemize}
      \item For stats, roll 4d6, drop lowest and assign at will.
      \item If any stat is a 3, 4, or 5 then it can be re-rolled.
      \item Starting gold is 4d6 drop lowest times 10.
      \item At level 1, PCs get max hit points.
      \item Humans get a +1 bonus to all saves.
\end{itemize}

\subsection{Advancement}
\begin{itemize}
      \item Hit point increments are rolled, but have a floor of half the hit
            dice value.
      \item On advancement, roll 1d6 for each stat. On a 6, that stat increases
            by 1.
      \item All classes advance at an equal rate. Since this negates the faster
            advancement of human races, and since they lack any of the special
            abilities that other races possess, humans get +2 to all saves.
      \item XP is not tracked by monsters slain or gold earned. Advancement is
            through overcoming and surviving hardships and trials. So long as
            something heroic is attempted by the party, one XP is awarded per
            session.
\end{itemize}
Advancement is according to the following table.
\input{scarlet_heroes_advancement.tex}

\section{Healing \& Death}
We are using the optional \emph{Negative Hit Points} rule (page 152).
\begin{itemize}
      \item At 0 HP PCs are dying. Any further damage will reduce their hit
      points to negative values.
      \item Dying characters lose an additional 1 hit point at the end of every
      round.
      \item If hit points become less than -10 then the character dies immediately.
      \item Before this point is reached, the character may have their wounds
      bound or receive magical healing. This will stabilise the character. A
      stable character does not lose 1 hit point each round.
      \item A dying or stable character cannot move more than a few feet without
      help, nor fight, nor cast spells until their hit points are again greater
      than zero.
      \item To treat a fallen character, a PC must spend their combat turn
            applying first aid or healing. They cannot attack on that turn.
      \item 8 hours of rest is required to restore 1 HP\@. A full day spent
            resting will restore 3 HP\@.
\end{itemize}

\section{Magic}
\begin{itemize}
    \item A Magic-User may learn a spell and transcribe it into their spellbook
    without a gold cost by succeeding on an \emph{Intelligence Check}. If they
    fail the check they must wait until reaching the next level to try again.
    Alternatively, they can spend a day studying the spell and incur a cost of
    100gp per spell level.
\end{itemize}

\section{Checks \& Saves}
The Ability Checks system from \emph{Labyrinth Lord} is used. To make a check, a
d20 must be rolled under the ability. Rolling high is better. Rolling exactly
the ability score is the best possible outcome. If conditions are particularly
favourable then the check is made with \emph{Advantage}. If conditions are very
unfavourable then the check is made with \emph{Disadvantage}.

\section{Combat}
\subsection{Critical Hits and Misses}
A natural 20 on an attack roll is a \emph{\textbf{Critical Hit}}. Critical hit
damage is calculated as \emph{max weapon damage + damage roll + bonuses}. If
the attack target is killed, then surplus damage can be applied to other
enemies in reach so long as they can be harmed by the attack and weapon.

A natural 1 on an attack roll is a \emph{\textbf{Critical Failure}}. On a melee
attack, the character is unable to attack on their next turn. Ranged attacks
may damage allies if they are in melee combat with the original target.